\documentclass{article}
\usepackage[utf8]{inputenc}	%legt den Zeichensatz der eingabedatei fest, (umlaute)
\usepackage{amsmath}				%Zusätzliche Funktionen für Formeln 
\usepackage{eurosym}
\usepackage{float}				%Bilder und Figuren plazierungen
\usepackage{graphicx}			%Bilder einbinden 
\usepackage{ngerman}				%Deutsche sprache/Datumsformat Inhaltsverzeichnis
\usepackage{hyperref}			%Links in der PDF Datei

\usepackage{dcolumn}				%Digitale Ziffern in Tabellenspalten 
\newcolumntype{d}{D{,}{,}{2}}

\setlength{\parindent}{0cm}

\begin{document}
\title{Endlich!}
\author{Hans Freitag (OE1SRC), Clemes Hopfer (OE1RFC)}
\maketitle
\begin{abstract}
Nach langer Wartezeit ist es endlich da das Runbo X5. In der QSP ausgabe vom Juni 2010 haben wir einen Wink mit dem Zaunpfahl gegeben, heute 
kann man es in China bestellen. Ein Mobiltelefon mit Touchdisplay, GPS, Bluetooth und $70cm$ Funkgerät, das offene Betriebssystem Android 
ermöglicht selbstprogrammierte Apps. Kostenpunkt ca $300,- \euro$. Wir haben uns das Gerät einmal genauer angesehen. 
\end{abstract}

\section{Erster Eindruck}

Ein riesiger trümmer. Das Telefon erinnert doch stark an alte Zeiten wo man noch extra Rucksäcke für die Telefone gebraucht hat. Nur am Display 
ist ersichtlich das das Telefon doch auch neuerer Zeit stammt, schließlich wollen aber alle features auch untergebracht werden. Insbesondere 
der $3,8 Ah$ Akku. Die Bedienung ist flüssig die verarbeitung sauber, die Funkgeräte app ist allerdings dazugebastelt. Die geht nur im Querformat, 
und eine Lautstärkeregelung des Funkgerätes ist ebenfalls nicht möglich. Die Befürchtung das der Hersteller so dumm war den Audioausgang vom 
Tranceiver Baustein direkt auf den Klingelton Lautsprecher zu legen wächst.\\

Ebenso benutzt der FM Radio empfänger das Kabel des Ohrhörers als Antenne, das ist zwar standard für Mobilfunkanwendungen, aber mit der 
eingebauten 70cm anenne ziemlich unsinnig. \\


\section{Hardware}

Da die Technischen Daten ziemlich mager ausfallen, erstmal eine zusammenfassung der vom Hersteller verbauten Hardware: 

\subsection{Android}

\subsection{Funkgerät}

Das 70cm Funkgerät basiert auf einem DRA808M\cite[dra808m] 30 dBm Wireless Voice Transceiver Module. Das Modul kann über einen Pegel 
wahlweise mit 1W oder 500mW senden und wird ansonsten über ein Set AT Komandos via UART gespeist. Zumindestens kann so schonmal die 
dahingeschriebene Steuerapp ersetzt werden. \\


\section{Links}

\begin{thebibliography}{websites}
    \bibitem{hersteller}Hersteller Webseite 
    \bibitem{dra808m}DRA808M Wireless Voice Tranceiver \url{http://www.dorji.com/docs/data/DRA808M.pdf}
\end{thebibliography}



\end{document}
